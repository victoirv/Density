\documentclass[10pt]{article}
\usepackage{graphicx}
\usepackage{subfigure}
\usepackage{fancyhdr}
\usepackage{amssymb,amsmath}
\usepackage{nicefrac}
\usepackage[usenames,dvipsnames]{color}
\usepackage[colorlinks,citecolor=RedViolet,urlcolor=blue]{hyperref}
\usepackage{doi}
\usepackage{setspace}
\usepackage[paperwidth=8.5in, paperheight=11in,top=2in, bottom=1.5in, left=1.2in, right=1.2in]{geometry}

\usepackage[authoryear,square]{natbib}
\pagestyle{fancy}
\fancyhead[R]{Veibell \thepage}
\fancyhead[L]{}

\begin{document}
\title{Geomagnetic Storm Precursors}
\author{Victoir Veibell\footnote{vveibell@gmu.edu}}
\maketitle
\hrule
\setlength{\parskip}{3ex}
\renewcommand{\labelitemi}{$-$}

\section{Prior work}
\citep{Takahashi} state that spikes in the Disturbance Storm Time (DST) index coincide with with significant changes in equatorial mass density. They look at five specific storms over a 20 day period with evidence that two had density spikes after the DST drop, two had density spikes before the drop, and one showed no effect. 

\section{Our work}
Our work takes the datasets from \citep{Denton} and \citep{Reconstruction}, uses a mean interpolation to take the mass density parameter from a highly non-uniform 15 minute grid of the former to the contiguous one hour grid of the latter, and looks for storms based on the definition of storm periods provided by \citep{Takahashi}, namely when $DST<-50nT$. The former has a time coverage of 1980-1991, but covered by different satellites in different years (sometimes with overlap), and the latter data set covers 1972-2013. By then looking at an hourly average of variables from 24 hours before storm onset to 48 hours after, trends in storms can be discerned. 


\bibliographystyle{vplainnat}

\bibliography{reportbib}

\end{document}