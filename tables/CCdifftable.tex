 \begin{table}[h]
 \small
 \begin{tabular}{|L|LLLL|}
 \hline
T-Tr & \text{GOES 2} & \text{GOES 5} & \text{GOES 6} & \text{GOES 7}\\ \hline
DoY & -0.16\pm0.11 & -0.02\pm0.16 & -0.12\pm0.08 & 0.01\pm0.12 \\
MLT & -0.26\pm0.25 & -0.16\pm0.14 & -0.16\pm0.26 & -0.11\pm0.07 \\
B_z & -0.06\pm0.24 & -0.27\pm0.18 & -0.05\pm0.17 & -0.14\pm0.08 \\
V_{sw} & -0.11\pm0.13 & -0.02\pm0.13 & -0.03\pm0.13 & -0.12\pm0.08 \\
D_{st} & -0.02\pm0.21 & -0.01\pm0.12 & -0.06\pm0.16 & 0.00\pm0.17 \\
\rho_{sw} & -0.01\pm0.29 & -0.01\pm0.37 & -0.04\pm0.22 & 0.03\pm0.21 \\
F_{10.7} & 0.01\pm0.11 & -0.04\pm0.16 & 0.00\pm0.08 & -0.03\pm0.08 \\
B_z+V_{sw} & -0.15\pm0.20 & -0.17\pm0.20 & -0.08\pm0.12 & -0.24\pm0.08 \\
D_{st}+F_{10.7} & -0.05\pm0.12 & 0.02\pm0.11 & -0.01\pm0.10 & -0.03\pm0.08 \\
All & -0.64\pm0.22 & -0.54\pm0.28 & -0.16\pm0.14 & -0.25\pm0.13 \\
 \hline
 \end{tabular}
 \caption{Table of differences in linear testing-training models, where each correlation is the median correlation of 100 random samples. Each sample trained on half of the data (via randomly selected rows of the least squares matrix) and tested on the other half} 
 \label{CCdifftable}
 \end{table}
