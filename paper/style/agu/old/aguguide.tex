
%%  AGUGUIDE.TEX  -- August 6, 1996 version
%
%%  User Guide for AGU's LaTeX Style Files.
%
%%  This document should be LaTeXed and 
%   printed, not viewed on screen.

\documentstyle[agupp]{article}
\setcounter{secnumdepth}{4}
\hyphenation{com-pu-scripts manu-script 
manu-scripts Springer-Ver-lag}

\begin{document}
\twocolumn

\title{\Large AGU's \LaTeX\ STYLE FILES Version 4.0}
\vspace{-2pc}

\begin{article}
\section{Introduction --- New Style Files}

\LaTeX\ style files have been created in order to 
ease the task of preparing AGU manuscripts for publication.
This guide contains instructions for using AGU's \LaTeX\ 
package, which consists of substyles to the standard \LaTeX\ 
{\tt article} style.

These instructions accompany version 4.0 of AGU's \LaTeX\ 
style files.  Version 4.0 contains many features and 
additions, such as automatic generation of the second 
set of figure captions and the potential to generate JGR 
supplemental abstracts.  Earlier formatting problems have 
been corrected as well.

Files created in previous versions of AGU's style files 
will run with this version, though you will need to include 
a \verb"\begin{article}" command in front of the body text 
and \verb"\end{article}" after the references (these 
control the single-column format).

Authors are expected to have at least a basic knowledge of 
\LaTeX.  Those who are unfamiliar with \LaTeX\ will need 
additional sources of information.  A number of publications 
are listed in the reference section of this guide.

AGU does not yet accept electronic submission of 
manuscripts.


\section{For Assistance}

For questions about AGU's \LaTeX\ style files, e-mail 
instruct@kosmos.agu.org (include your AGU paper number, 
if available) or call 202/462-6900, ext. 216.

To ask about the status of a manuscript, contact 
AGU's Author Information Line at 202/939-3200 
or e-mail cust\_ser@kosmos.agu.org (include your 
AGU paper number, if available).


\section{Creating Camera-Ready Manuscripts in \LaTeX}

Commands needed to create a camera-ready manuscript will 
be described in the same order they should appear in the 
input file.

The sample.tex file that accompanies this guide can be 
used as a sample document (simply \LaTeX\ and print it) 
or it can be used as a template (replace the sample text 
with the text of your manuscript and delete any extraneous
commands).


\subsection{Selecting a Document Style}

The first command in a \LaTeX\ manuscript must declare 
the overall style of the document.  AGU's \LaTeX\ 
style files provide several different formats.  There are 
five ``galley'' styles, a ``manuscript'' style, and a 
``preprint'' style.

The galley styles create camera-ready galley pages that 
authors may submit to AGU for publication.  Note that 
only one column is printed on a page; the single columns 
are placed into two-column format by AGU staff.
\vspace{10pt}

$\bullet$ The {\tt jgrga} substyle creates camera-ready 
          manuscripts for three journals:  {\it Journal of 
       Geophysical Research}, {\it Global Biogeochemical 
      Cycles}, and {\it Paleoceanography}.
\vspace{10pt}

$\bullet$ The {\tt grlga} substyle creates camera-ready 
       manuscripts for {\it Geophysical Research Letters}.
\vspace{10pt}

$\bullet$ The {\tt tecga} substyle creates camera-ready 
        manuscripts for {\it Tectonics}.
\vspace{10pt}

$\bullet$ The {\tt radga} substyle creates camera-ready 
   manuscripts for {\it Radio Science}.
\vspace{10pt}

$\bullet$ The {\tt rtjga} substyle creates camera-ready 
       manuscripts for the Russian translation journals.
\vspace{10pt}

$\bullet$ The {\tt agupp} substyle (preprint substyle) 
   creates a two-column format suitable for 
       distribution among colleagues.  This style can 
         be used as is, or \LaTeX\ style writers at your 
        institution may use this as a starting point for 
       creating their own preprint styles.  Do not submit 
     output from this style to AGU as camera-ready copy!
     Please note that the two-column format begins 
          at the point a \verb"\twocolumn" command appears 
       in the text.  If \verb"\twocolumn" is not 
      specified, the introductory material of the paper 
      will be set in one-column mode and the two-column 
      mode will engage with the first \verb"\section" 
        command.
\vspace{10pt}

$\bullet$ The {\tt agums} substyle (manuscript substyle)
   creates pages with large, double-spaced type with 
      wide margins for first-time submission to AGU 
          editors (i.e., for peer review and copyediting).
        This style uses a 12-point font.  If a font size 
       smaller than 12 is chosen, the {\tt agums} substyle 
    issues a warning message and will automatically 
        reset the size to 12 points; the file will still 
       be processed.
\vspace{10pt}

These style files are all substyles to the standard 
\LaTeX\ {\tt article} style and should be selected 
as substyle options, as shown:

\begin{center}
\begin{quote}
\verb"\documentstyle[jgrga]{article}" \\
\verb"\documentstyle[grlga]{article}" \\
\verb"\documentstyle[tecga]{article}" \\
\verb"\documentstyle[radga]{article}" \\
\verb"\documentstyle[rtjga]{article}" \\
\verb"\documentstyle[agupp]{article}" \\
\verb"\documentstyle[12pt,agums]{article}"
\end{quote}
\end{center}

\noindent The \verb"\documentstyle" command must appear 
first in any \LaTeX\ file.  Only one \verb"\documentstyle" 
may be chosen; other \verb"\documentstyle" commands must 
either be deleted or preceded by a comment symbol~({\tt \%}).


\subsection{Running Heads}

Authors must supply running head information.
There are two running head commands: \verb"\lefthead{}" 
and \verb"\righthead{}".

In the \verb"\lefthead{}" command, type only the last 
name(s) of the author(s) in all capital letters between 
the curly braces.  If your manuscript has three or 
more authors, type only the last name of the first author 
followed by ``ET AL.''

In the \verb"\righthead{}" command, type a short version of 
the article title (or include the entire title, if it fits) 
in all capital letters between the curly braces.

You may type a total of 60 characters (including letters, 
punctuation, and spaces) between the curly braces in 
either the \verb"\lefthead" or \verb"\righthead" commands 
(i.e., 10 characters in one command and 50 in the other 
is acceptable).  Please avoid nonstandard abbreviations.


\subsection{Received, Revised, and Accepted Dates}

\begin{quote}
\verb"\received{January 13, 1996}"\\
\verb"\revised{April 3, 1996}"\\
\verb"\accepted{August 6, 1996}"
\end{quote}

Received, revised, and accepted dates are supplied by 
your AGU editor.  The information typed in these commands 
will print at the end of the manuscript (after the 
references and author addresses).  Please type dates 
in the format shown above and in the sample.tex document.

The {\tt agums} style prints a blank line for the 
{\tt received} date if no text is included.


\subsection{Manuscript Information}

{\it Geophysical Research Letters} (GRL) manuscripts often 
have such quick turnaround times that authors do not 
receive manuscript information for their documents.
If you are submitting a camera-ready manuscript to GRL,
you may leave the \verb"\paperid", \verb"\cpright",
and \verb"\ccc" commands empty.


\subsubsection{Paper number}

Type your manuscript's AGU paper number between 
curly braces in the \verb"\paperid{}" command.  All 
manuscripts accepted by AGU are assigned a paper number, 
which is used to track and identify manuscripts.  AGU 
paper numbers contain two digits indicating the year a
manuscript was received at AGU, two letters indicating 
the journal, and five digits indicating in what order 
the manuscript was received at AGU headquarters.


\subsubsection{Copyright}

Type your copyright information between the first
set of curly braces in the \verb"\cpright{}{YEAR}" 
command (where {\tt YEAR} is the year the manuscript
is published).  There are three copyright options 
available.
\vspace{11pt}

1.~~If your paper is AGU copyright, choose 
the  ``AGU'' \verb"\cpright" command.
\vspace{11pt}

2.~~If your paper is in the public domain, 
choose the ``PD'' \verb"\cpright" command.
\vspace{11pt}

3.~~If your paper is Crown copyright, 
choose the ``Crown'' \verb"\cpright" command.
\vspace{11pt}

Copyright type is case-sensitive, so you must
type the command exactly as shown above.

If you are not sure which copyright to choose, 
please contact your production coordinator.
You must choose one of these copyright options 
(except for GRL manuscripts).


\subsubsection{Copyright Clearance Center Code}

Type your Copyright Clearance Center code between
curly braces in the \verb"\ccc" command.  (This
code is provided by your journal editor.)  Crown 
copyrights have no \verb"\ccc" information.  Do not 
forget to place a backslash in front of the dollar
sign or it will be interpreted as a math command.


\subsection{Author Addresses}

For author postal addresses, group authors by affiliation 
and list them in alphabetical order between curly braces 
in \verb"\authoraddress" commands.

Names include first initial only, optional middle initial, 
and last name.  If available, include e-mail addresses as 
shown in the sample.tex document.  E-mail addresses may be 
broken after a period or after the @ sign.  If your e-mail 
address includes an underscore, don't forget to place a 
backslash in front of it or it will be interpreted as a 
command.

Note that author addresses are different from author
affiliations.  Author addresses should include complete
postal address information, while author affiliations
include only department, institution, town, and state
(if state is not included in the name of the institution)
or country (if not the United States).

When {\LaTeX}ed, author addresses appear after the
references section.


\subsection{Slug Comments}

Authors may include short remarks on the title page 
of {\tt agums} and {\tt agupp} substyles (e.g.\ the 
name and date of the journal in which an article 
is scheduled to appear).

Type slug comments between the curly braces of a
\verb"\slugcomment{}" command.

In the {\tt agums} style, slug comments appear after 
the manuscript dates; in the {\tt agupp} style, they 
appear in the upper left corner of the title page.
Slug comments do not appear in the camera-ready styles.


\subsection{Numbering Your Sections}\label{numbring}

If you want \LaTeX\ to automatically number your sections, 
type \verb"\setcounter{secnumdepth}{4}" in the preamble 
of your input file (i.e., in front of your 
\verb"\begin{document}" command, as shown in the 
sample.tex document).  If you do not want section numbers,
type \verb"\setcounter{secnumdepth}{0}" in the \linebreak
preamble of your input file.


\subsection{``Begin Document'' and Front Matter \protect\\
Commands}
  
The main portion of the manuscript starts with a
\verb"\begin{document}" command, which is followed
by the front matter (title, author(s), affiliation(s),
and abstract).


\subsection{Title}

Type the title of your manuscript between the curly 
braces in the \verb"\title{}" command.  Capitalize 
only acronyms, first letter of the first word, first 
letter of proper nouns, first letter of the first word
after colons, and first letter of the first word of a 
subtitle.

If the title exceeds one line, break it so that the 
first line is longer than the second line; break the 
title before articles, prepositions, and conjunctions.
To break the title, type a double backslash where you 
want the break to occur, as shown in the sample.tex 
document.


\subsection{Authors and Affiliations}

\subsubsection{Authors}

Type author names between the curly braces in 
\verb"\author{}" command(s).  Author names consist 
of first name or initial, optional middle name or
initial, and last name.

If it is necessary to break author lines, make sure 
you break them between two author's names (using 
double backslash \verb"\\" commands).


\subsubsection{Affiliations}

Type affiliation information between the curly braces 
in an \verb"\affil{}" command immediately after each 
\verb"\author" command.  \linebreak Each \verb"\author" 
command must be followed by a corresponding \verb"\affil"
command.


\subsubsection{Alternate affiliations}

Authors often have affiliations in addition to their 
principal employer.  If your authors have such 
alternate affiliations, type an \verb"\altaffilmark{}" 
command after the author name, type a footnote number 
between the curly braces, then type the corresponding 
alternate affiliation information between curly braces 
in an \verb"\altaffiltext{}{}" command below the primary
author affiliation commands, as shown in the sample.tex 
document.  There must be a separate \verb"\altaffiltext"
command for each alternate affiliation.

This information will print as footnotes after the 
reference section in your {\LaTeX}ed manuscript and 
will appear on the bottom left-hand corner of the 
first page of your published manuscript.


\subsubsection{Footnoted affiliations}

If your manuscript contains more than three author 
and affiliation groupings, then all author names must 
be typed together in a single \verb"\author" command.
Then, type an \verb"\altaffilmark{}" command after each 
author name, and type each author affiliation between 
curly braces in \verb"\altaffiltext{}{}" commands.

Each affiliation should appear in a separate \linebreak
\verb"\altaffiltext{}{}" command, and the number 
in the \verb"\altaffilmark{#}" command should match 
the number in the appropriate \verb"altaffiltext{#}{}" 
command.  There is an example of this format in the 
sample.tex document.


\subsection{Abstract}

Type your abstract between \verb"\begin{abstract}"
and \verb"\end{abstract}" commands.

An abstract consists of a one-paragraph summary
of your paper, in 250 words or fewer.  Do not cite
references unless absolutely necessary.  If you
must cite, place the citation in square braces
using italic type (this abstract citation format
is new).  Do not include displayed material.


\subsection{Sections}

AGU's \LaTeX\ style files support four section~levels.

$\bullet$~Level 1 heads use \verb"\section{}" commands.

$\bullet$~Level 2 heads use \verb"\subsection{}" commands.

$\bullet$~Level 3 heads use \verb"\subsubsection{}" commands.

$\bullet$~Level 4 heads use \verb"\subsubsubsection{}" commands.

For level 1 and level 2 heads, capitalize the first 
letter of each word (except for prepositions, conjunctions, 
and articles that are 3 letters or shorter).  Do not 
hyphenate level 1 or level 2 heads.  To break lines, 
type \verb"\protect\\" where you want a break to occur.

For level 3 and level 4 heads, capitalize only acronyms, 
the first letter of the first word, first letter of proper 
nouns, and first letter of the first word after colons.
Hyphenation is permitted in level 3 and level 4 heads, 
if needed.  You must include at least two level 3 heads 
per level 2 head.  A level 4 head cannot directly follow 
a level 3 head; there must be a least one sentence between
the two heads.

Note that these commands delimit sections by marking the 
beginning of each section; there are no separate commands 
to identify the ends.

If you want to number your sections, see the information 
in section \ref{numbring}, above.

\subsection{Manuscript Text}

Type a \verb"\begin{article}" command, then begin 
the text of your manuscript.  Include section heads
wherever appropriate.  You may begin your text with 
an introductory section head, but this is not required.

The body of the manuscript text must start with the 
\verb"\begin{article}" command, and you must include 
an \verb"\end{article}" command after the references 
section (otherwise, your text will not print at the 
proper column width).

Do not underline text.  Italics should only be used 
for math variables and text citations.

See section \ref{formtext} for information on formatting
text in \LaTeX.


\subsection{Appendices}

If your manuscript contains appendices, type an
\verb"\appendix" command, then type your appendix
section headers within \verb"\section{}" commands,
with the appropriate appendix introduction, for example,
\verb"\section{Appendix A: Your Title}."

If you have several appendix sections, lettering in
appendix headers should match the above example.
If there is only one appendix, lettering does not
appear in the header: \verb"\section{Appendix: Your Title}."

The \verb"\appendix" command causes all following equations 
to use a lettered number (for instance, A1).  This is to 
distinguish appendix equations from equations that appear 
in the main body of the manuscript.

Appendix tables must be manually labeled.  If your manuscript 
contains tables that accompany an appendix section, you must 
use \verb"\tablenum{}" commands and label the table(s) accordingly.
(If there is one appendix section and one table, the table 
number will be \verb"\tablenum{A1}"; if there are two appendix 
sections, tables appearing in the second section will begin 
with \verb"\tablenum{B1}", etc.)  See the sample.tex document for
examples.


\subsection{Acknowledgments and Editor \protect\\
Acknowledgments}

Authors may include an acknowledgments section, if 
desired.  Acknowledgments appear after the last paragraph 
of manuscript text or after the appendices (if any).
Acknowledgments may be only one paragraph in length.

Type an \verb"\acknowledgments" command, then begin your 
acknowledgments text.  Be sure to spell ``acknowledgments'' 
exactly as it appears; otherwise, \LaTeX\ will not recognize 
the command.  There is no ``end acknowledgments'' command.

If your manuscript contains only one acknowledgment, use 
an \verb"\acknowledgment" command to produce a singular 
``Acknowledgment'' subhead.

If your manuscript contains Editor's acknowledgments (for
{\it JGR--Space Physics} only) set them in a new paragraph 
one empty line below the regular acknowledgments, as shown 
in the sample.tex document.

\subsection{References and Citations}

There are two methods for managing citations and 
references.  One method uses \verb"\begin{references}"
and \verb"\end{references}" commands.  The second \linebreak
method uses \verb"\begin{thebibliography}{}" and linebreak
\verb"\end{thebibliography}{}" commands.  See the 
sample.tex document for examples of each style.


\subsubsection{Method 1}
Authors may enter properly formatted citations directly 
in the manuscript text and enclose those citations in 
\verb"\markcite{}" commands.  This method uses a 
\verb"\begin{references}" and \verb"\end{references}" 
environment for your reference section.  Method 1 
marks all citations in your manuscript, but there 
is no interaction between the \verb"\markcite" 
commands and the reference section.

Type a \verb"\begin{references}" command
to start your reference section.  This command
automatically produces a correctly formatted 
``Reference'' head.  Next, type a \verb"\reference" 
command.  This command must appear in front of 
every reference in your reference section, or 
your references will not format properly.
Type reference information after the 
\verb"\reference" command.  It is the author's
responsibility to place bibliographic reference
information in the proper order with correct 
punctuation.  After the last reference in your
reference section, type an \verb"\end{references}"
command, then type an \verb"\end{article}" command.

To create in-text citations, enclose each citation
within a \verb"\markcite" command.  In-text
citation format italicizes all author names
and uses nonitalic type for publication years.
There are two ways to include in-text citations,
depending on the way you phrase your sentence.
You may either include an entire reference within
square braces \markcite{[{\it Merritt et al.,} 
1996]} or you may mention the author as part of 
your sentence and include only the year in braces, 
as in \markcite{{\it Ono} [1996]}.  Method 2 cannot 
create the second citation format, so if your 
manuscript ever contains this format, you should 
probably use method 1.

\subsubsection{Method Two}
Authors may create automated in-text citations 
with \verb"\cite" commands that correspond to tags 
in \LaTeX's \verb"\thebibliography{}" environment.
This method requires authors to create proper 
journal format within \verb"\bibitem" commands.

Type a \verb"\begin{thebibliography}{}" command to start
your reference section.  This command automatically 
produces a correctly formatted ``Reference'' head.

Next, type \verb"\bibitem[LABEL]{TAG} \reference"
where {\tt LABEL} is the correct AGU in-text
citation format (i.e.,
\verb"\bibitem[{\it Johnson,} 1996]{j96}" \linebreak
\verb"\reference").  {\tt TAG} text is chosen by the 
author and may follow any preferred format.  Type 
the full text of your reference after the 
\verb"\bibitem" command.  A \verb"\bibitem" 
command must appear in front of every reference 
in your reference section, or your references will 
not format properly.  It is the author's responsibility
to place bibliographic reference information in the 
proper order with correct punctuation.  After 
the last reference in your reference section, 
type an \verb"\end{thebibliography}" command, 
then type an \verb"\end{article}" command.

To create in-text citations, type \verb"\cite{TAG}", 
where the {\tt TAG} is identical to the {\tt TAG} 
used in the corresponding reference's 
\verb"\bibitem[]{TAG}" command.  (For example,
\verb"\cite{j96}".)

If you cite two references from two different years,
you must type only one \verb"\cite" in the input file 
and include the second year in square braces, as shown
in the sample.tex document, in order to output correct 
AGU reference style.

If you cite two references in the same year at
the same time, you must use an ``a, b'' format
after the publication year and code the 
\verb"\cite" command in the text as shown 
in the sample.tex document.


\subsubsection{AGU reference style}
Examples of a variety of publications referenced
in AGU bibliographic style are included in the
sample.tex document.  Contact AGU's Editorial
Services Department if you have a question about 
AGU reference style.


\subsubsubsection{In press}
AGU gives a full reference for works accepted for 
publication but not yet released by using the words 
``in press'' in place of the journal volume and 
page numbers and by using the year of acceptance as 
the date.  Manuscripts submitted for publication 
but not yet accepted are considered to be unpublished.


\subsubsubsection{Unpublished work}
AGU does not include personal communications, unpublished 
data, and manuscripts in preparation or submitted for 
publication in the reference list because AGU does not 
consider them to be part of public literature.  Refer to 
these works in text in parentheses by first initials and 
last name of source, type of material, and date.  For 
submitted papers include title and journal submitted to, 
if available.  If you wish, you may include the affiliation 
of the source, for example (J. Wilson, personal communication, 
1971) or (B. L. Smith, University of Massachusetts, unpublished 
data, 1979).


\subsubsubsection{Short journal commands}
Short commands have been created for some of the 
journals referenced most often.  Authors may use 
these commands in reference sections as shorthand 
rather than typing out the journal name.  Note 
that a comma is automatically included after each 
journal listing.

To avoid confusion, the following list shows full 
journal names, but the short commands will produce 
abbreviations appropriate for AGU bibliography 
style (thus ``\verb"\apj"'' produces ``\apj''):
\vspace{-10pt}
\begin{center}\begin{tabular}{ll}
\verb"\apj"    & {\it Astrophysical Journal}\\
\verb"\bams"   & {\it Bulletin of the American Meteorological}\\
               & \hspace{2em}{\it Society}\\
\verb"\bssa"   & {\it Bulletin of the Seismological Society of}\\
               & \hspace{2em}{\it America}\\
\verb"\dsr1"   & {\it Deep Sea Research Part I}\\
\verb"\dsr2"   & {\it Deep Sea Research Part II}\\
\verb"\eos"    & {\it Eos (Transactions of the American}\\
               & \hspace{2em}{\it Geophysical Union)}\\
\verb"\epsl"   & {\it Earth and Planetary Sciences Letters}\\
\verb"\gca"    & {\it Geochimica and Cosmochimica Acta}\\
\verb"\gji"    & {\it Geophysical Journal International}\\
\verb"\gjras"  & {\it Geophysical Journal of the Royal}\\
               & \hspace{2em}{\it Astronomical Society}\\
\verb"\grl"    & {\it Geophysical Research Letters}\\
\verb"\gsab"   & {\it Bulletin of the Geological Society of}\\
               & \hspace{2em}{\it America}\\
\verb"\jatp"   & {\it Journal of Atmospheric and Terrestrial}\\
               & \hspace{2em}{\it Physics}\\
\verb"\jgr"    & {\it Journal of Geophysical Research}\\
\verb"\jpo"    & {\it Journal of Physical Oceanography}\\
\verb"\mnras"  & {\it Monthly Notices of the Royal}\\
               & \hspace{2em}{\it Astronomical Society}\\
\verb"\mwr"    & {\it Monthly Weather Review}\\
\verb"\pag"    & {\it Pure and Applied Geophysics}\\
\verb"\pepi"   & {\it Physics of the Earth and Planetary}\\
               & \hspace{2em}{\it Interiors}\\
\verb"\pra"    & {\it Physical Review A: General Physics}\\
\verb"\prb"    & {\it Physical Review B: Solid State}\\
\verb"\prc"    & {\it Physical Review C: Nuclear Physics}\\
\verb"\prd"    & {\it Physical Review D: Particles and}\\
               & \hspace{2em}{\it Fields}\\
\verb"\prl"    & {\it Physical Review Letters}\\
\verb"\qjrms"  & {\it Quarterly Journal of the Royal}\\
               & \hspace{2em}{\it Meteorological Society}\\
\verb"\rg"     & {\it Reviews of Geophysics}\\
\verb"\rs"     & {\it Radio Science}\\
\verb"\usgsof" & {\it U.S. Geological Survey Open File}\\
               & \hspace{2em}{\it Report }\\
\verb"\usgspp" & {\it U.S. Geological Survey Professional}\\
               & \hspace{2em}{\it Papers}\\
\end{tabular}
\end{center}


\subsubsection{No reference section}
Both the \linebreak
\verb"\end{references}" and 
\verb"\end{thebibliography}{}" \linebreak
commands cause manuscript information to print.
If your manuscript contains no reference section, 
you must place a \verb"\forcesluginfo" command in the 
manuscript after the last paragraph of text (or 
after the acknowledgments, if any).


\subsection{Equations}

AGU journal style allows two types of equations: 
in-text equations and displayed equations.

In-text equations occur within a normal paragraph of
text.  These equations must not interfere with line
spacing.  If an in-text equation does affect line
spacing, it must be coded as a displayed equation.

Displayed equations are single-line or multiline
equations with extra space between the equation
and the preceding and following paragraphs of text.


\subsubsection{In-text equations}

To create math in normal text, type the 
equation within single dollar signs (\$), 
thus \verb"$\pi r^2$" yields $\pi r^2$.


\subsubsection{Displayed equations}

\LaTeX's dis-\linebreak played math environments 
allow displayed equations to be typeset in many 
ways.  The following three are probably used most.


\subsubsubsection{Unnumbered displayed equations}

To create unnumbered displayed equations, type your
equation between \verb"\begin{displaymath}" and  \linebreak
\verb"\end{displaymath}" commands (or use the 
shorter equivalent commands, \verb"\[" and \verb"\]").


\subsubsubsection{Numbered displayed equations}

To create numbered displayed equations, type your
equation between \verb"\begin{equation}" and \linebreak
\verb"\end{equation}" commands.


\subsubsubsection{Aligned and multiline displayed equations}

To create multiline equations or to vertically align
an equation (for example, a derivation where alignment 
is wanted on the equal sign), type \linebreak 
your equation between \verb"\begin{eqnarray}" and \linebreak
\verb"\end{eqnarray}" commands.


\subsubsection{Numbering equations}

By default, \LaTeX\ will number equations sequentially 
from the beginning to the end of a manuscript.  If you 
must specify equation numbers, use \verb"\eqnum{#}", 
where \# is the appropriate equation number, inside 
an {\tt equation} or {\tt eqnarray} environment.
\LaTeX's equation counter is not incremented when 
\verb"\eqnum" is used.

When unnumbered equations are desired, use either the 
{\tt displaymath} environment (for single-line displayed
equations) or use a \verb"\nonumber" command with an 
{\tt eqnarray} environment, (for multiline equations).
The \verb"\nonumber" command must be placed before the 
double backslash (\verb"\\").  \LaTeX's equation counter
is not incremented when \verb"\nonumber" is used.

If the use of \verb"\eqnum" or \verb"\nonumber" causes 
\LaTeX's equation counter to number equations in the wrong 
sequence, the counter may be reset with a \linebreak
\verb"\setcounter{equation}{#}" command, where \#
is the next desired equation number minus one (if
the next number is 7, \# should be 6).  This 
command must be used outside the equation environments.


\subsubsection{Lettering equations}

It may be necessary to group related equations together 
and identify them with letters appended to the same 
equation number (as opposed to each equation having a 
separate numeral).  Set such related equations in 
{\tt equation} or {\tt eqnarray} environments 
(whichever is appropriate may be used), and place 
this grouping within a {\tt mathletters} environment.
See the sample.tex document for examples.


\subsubsubsection{Keyboard math symbols}

You may create some math symbols simply by typing the 
appropriate keyboard command, but most of these commands 
must appear within a math or equation environment or they 
will not print correctly.\\
\verb" +  -  =  <  >  /  :  !  '  |  [  ]  (  )"

%\begin{displaymath}~~ + ~~ - ~~ = ~~ < ~~ > ~~ / ~~ : ~~ ! ~~ ' ~~ | ~~ [ ~~ ] ~~ ( ~~ ) ~~ \end{displaymath}


\subsubsubsection{Fractions in displayed equations}

AGU's \LaTeX\ style files contain an alternate command 
for fractions.  It may be unnecessary to use any command 
other than the standard \LaTeX\ \verb"\frac" command.
\LaTeX\ will set fractions in displayed math as built-up 
fractions.  However, it is sometimes desirable to use 
case fractions in displayed equations.  In such instances, 
authors may a \verb"\case" command rather than the 
\verb"\frac" command.  A shilled fraction is produced 
without any special markup, as shown:
\vspace{-4pt}
\begin{displaymath}
\renewcommand{\arraystretch}{1.4}
\begin{array}{llc}
\mbox{Built-up}    & 
\verb"\frac{1}{2}" & 
\displaystyle\frac{1}{2}\\[.5ex]
\mbox{Case}        & 
\verb"\case{1}{2}" & 
\case{1}{2} \\
\mbox{Shilled}     & 
\verb"1/2"         & 
1/2 \\
\end{array}
\end{displaymath}
\vspace{-10pt}

\subsection{Tables}

AGU's \LaTeX\ style files support two table mechanisms:
(1) A {\tt planotable} environment that facilitates the 
formatting of lengthy tables, and (2) \LaTeX's 
standard {\tt table} and {\tt tabular} environments.
Short tables (smaller than one manuscript page) may 
be created using either mechanism; tables longer than 
one page will require the use of {\tt planotable}.

It is possible to create fairly complex tables with 
arbitrary spacing, straddle heads, rules, etc. in \LaTeX.
Authors who need to create complicated tables should 
consult the \LaTeX\ manual [{\it Lamport}, 1985] for 
details.  Most of \LaTeX's {\tt tabular} capabilities 
are applicable to {\tt planotables} as well.


\subsubsection{Planotables}\label{tbl-sec}

Planotables automatically produce the table lines and
vertical table spacing required for correct AGU table style.
In addition, planotables have several capabilities that 
facilitate the formatting of tables.  For instance, 
it is possible to break long planotables across pages, 
and, where tabular tables print at an automatic width,
authors may choose a specific planotable width.

Type \verb"\begin{planotable}{COLS}", 
where {\tt COLS} \linebreak 
sets the justification for each column.
Choose one letter (``l,'' ``c,'' or ``r'') for each 
column, indicating left, center, or right justification.

To set a planotable to a specific width, type \linebreak
\verb"\tablewidth{}", with the desired table width between 
the curly braces, after the \verb"\begin{planotable}" 
command.  For instance, to create a single-column table 
for the {\it Journal of Geophysical Research,} type 
\verb"\tablewidth{20pc}", where ``20pc'' is 20 picas.
AGU journals permit a variety of table widths.  See the 
``Table Data'' information in your General Instruction 
booklet for the table widths needed for your journal.

Type planotable titles within the curly braces of
\verb"\tablecaption{}" commands.  Capitalize the first
letter of each word (except for prepositions, conjunctions,
and articles that are three letters or shorter).  Do not
allow table caption lines to hyphenate; use a
\verb"\protect\\" command to break lines where necessary.
The \verb"\tablecaption" command automatically generates
the ``{\bf Table \#.}'' information.  To change table
numbers, see ``numbering tables'' in section \ref{numtabl}.

Type a \verb"\startdata" command, and then type your table
data.  The startdata command formats column headings, 
engages the tabular formatting, and produces the table 
caption.  Data within a table row are separated by {\tt \&} 
(ampersand) characters.  The end of each row is indicated 
with a \verb"\nl" command.  Extra vertical space can be 
inserted between rows with a \verb"\vspace{}" command 
(type the desired amount of space between the curly
braces).

If a planotable is longer than one page, \LaTeX\ will 
automatically break it across pages.  To force a page 
break in a particular place, type a \verb"\tablebreak" 
command.  This command affects the following line of 
table data (not the line it appears with) as shown 
in the sample.tex document.

If a cell contains no data, type a \verb"\nodata" 
command to create a ``no data'' symbol.


\subsubsection{Tabular tables}

Tabular tables must begin with a \verb"\begin{table}" 
command and end with an \verb"\end{table}" command.
This {\tt table} environment must enclose the table 
caption, all tabular material, and any footnote or 
table note information associated with a particular table.

Type tabular table titles within the curly 
braces of \verb"\caption{}" commands.  Capitalize 
the first letter of each word (except for prepositions, 
conjunctions, and articles that are three letters or 
shorter).  Do not allow table caption lines to 
hyphenate; use a \verb"\protect\\" command to break 
lines where necessary.  The \verb"\caption" command 
will automatically generate the ``{\bf Table \#.}'' 
information.  To change table numbers, see 
``numbering tables'' in section \ref{numtabl}.

To begin table data, type \verb"\begin{tabular}{COLS}" 
where {\tt COLS} sets the justification for each column.
Choose one letter (``l'', ``c'', or ``r'') for each column,
indicating left, center, or right justification.  Consult the 
\LaTeX\ manual [{\it Lamport}, 1985] for details about using 
a {\tt tabular} environment to prepare tables.  Include only 
one {\tt tabular} environment within each {\tt table} environment.
Type \verb"\end{tabular}" after the table data (but before the 
\verb"\end{table}" command.

Authors must use \verb"\tableline" and spacing commands 
within tabular tables in order to create correct AGU 
table style.  The \verb"\tableline" command produces
a horizontal rule and must be included after the caption,
after the column headings, and after the table data (but
before any table footnotes or table comments).  Set at 
least 6 points of space between each rule and each line 
of text, including any table notes or footnotes.  AGU style
does not permit vertical rules.  Use either \verb"\vspace{}" 
or brace commands (such as \verb"\\[4ex]") to insert 
space.  Examples of both appear in the last table of the 
sample.tex document.


\subsubsubsection{Notation tables}

If your manuscript contains notation tables, use a level 
1 head for the word ``Notation'' and use a tabular 
environment without any rules for the notation information,
as shown in the sample.tex document.  If you numbered your 
sections, place a \verb"\setcounter{secnumdepth}{0}" in 
front of the Notation section head, so it will appear 
flush left and without numbers.  Set both notation
columns flush left.


\subsubsection{Table column headings}

Table column headings must be enclosed in a \verb"\tablehead" 
command.  The individual headings then appear within either 
\verb"\colhead" or \verb"\multicolumn" commands.  Each column 
of data must have a heading.  The double backslash command 
(\verb"\\") may be used to end table headings, and extra 
vertical spacing can be inserted with brace commands 
(\verb"\\[.5ex]").

It is possible for a complicated table heading to overflow
the vertical space allotted for the table heading.  The
fraction of the page allocated for the table heading may be
changed with \verb"\tableheadfrac".  The {\tt NUM} argument
to \verb"\tableheadfrac" should be the decimal fraction of
the page used for heading information.  The default value
is 0.1, meaning that 10\% of the page height is reserved
for the table heading.  It should rarely be necessary to
change this value.


\subsubsection{Table footnotes and table comments}

If your table contains material requiring footnotes, use a
\verb"\tablemark{TAG}" command to indicate the footnote,
then type the associated information (with a corresponding
{\tt TAG}) within a \verb"\tablenotetext{TAG}{}" command.

Short table comments can be created using a \linebreak 
\verb"\tablenotetext{\null}{TEXT}" command, with a \linebreak
\verb"\null" argument as the tag.  Longer comments may
be placed within a \verb"\tablecomment{TEXT}" command.
Lists of table references should use the following
format: \verb"\tablecomment{References:  Names of your
references.}"  Only one paragraph of material is permitted 
at the end of a table, (excluding superscripted footnotes),
so if both references and notes exist, they should
be run in together.  See the sample.tex document for 
examples of table footnotes and comments.

Both \verb"\tablenotetext" and \verb"\tablecomment" 
commands must appear after the column heading information
and before the table data; otherwise, they will affect
indentation of the last table cell.  These commands 
work with planotables and tabular tables.

\subsubsection{Numbering tables}\label{numtabl}

To choose a specific table number or to add a
letter before or after a table number, type a 
\verb"\tablenum{}" command on the line after a 
\verb"\begin{table}" or \verb"\begin{planotable}" 
command.  Type the desired table number between 
the curly braces in the \verb"\tablenum{}" command.
\linebreak \LaTeX's equation counter is affected 
when \verb"\tablenum" commands are used.

\subsubsection{Centered table heads}
AGU's \LaTeX\ style files allow centered, italic 
headings within planotables.  Type center head 
text within a \verb"\cutinhead{}" command.  At 
least two center heads per table are required 
if any are used.


\subsection{Figures and Plates}

Do not incorporate figure or plate captions
into the main text of your manuscript.  All captions 
should appear at the end of your submitted manuscript 
package.  Captions will be placed with the appropriate 
artwork when AGU staff prepares camera-ready pages.

Use a figure environment to create captions for 
gray scale or black-and-white line art; use a plate 
environment to create captions for any color artwork.

See the General Instruction booklet for information on
submitting artwork.


\subsubsection{Figures}

Type \verb"\begin{figure}", type your figure caption 
text between the curly braces in a \verb"\caption{}" 
command, then type \verb"\end{figure}".  Each separate 
figure caption must have at least three commands: a
\verb"\begin{figure}" command, a \verb"\caption" command, 
and an \verb"\end{figure}" command, as shown in the 
sample.tex document.

Several captions may be printed per page, as long 
as there is sufficient room to cut between the captions.

It may be necessary to use \verb"\clearpage" commands 
between a long series of captions, since too many 
captions can cause \LaTeX's memory buffer to overload 
and crash.  The \verb"\clearpage" command clears out 
the memory buffer and begins a new page.

AGU asks authors to submit two sets of captions; one 
set for single-column and one set for double-column 
figures.  You must type in the first set of captions, 
but AGU's \LaTeX\ style files automatically generate 
a second set of captions.  Captions print at the 
correct single- and double-column widths for your 
journal (these widths change depending on the 
document style you choose).

If you need to change the width of a figure caption
(for instance, to create broadside captions), type a 
\verb"\figurewidth{}" command, with the desired
figure width in the curly braces.  When used inside 
a figure environment, the \verb"\figurewidth" command sets 
the width of the first set of captions to the specified 
width.  When used in front of a series of figure captions,
the \verb"\figurewidth" command affects all following
caption widths.  Note that \verb"\figurewidth" commands 
do not affect the automated second set of captions.

Authors may type a \verb"\figurenum{}" command after
a \verb"\begin{figure}" command to explicitly number a
figure caption.  Letters or symbols may also be included.
This command does affect \LaTeX's figure counter.  Figure 
numbers of the second set of captions will always mirror 
the numbers of the first set of captions.  To reset 
\LaTeX's figure counter, type a \verb"\setcounter{figure}{#}" 
command in front of a \linebreak
\verb"\begin{figure}" command.  Set the \# to zero if 
the next desired figure number is one, or to three if the
next desired figure number is four, etc.  This command 
affects figure numbers of any following figures.

Figure caption lines may be broken by using a 
\verb"\protect\linebreak" command.

Be aware that figure width numbering commands may affect 
following plate captions and vice versa.


\subsubsection{Plates}

The plate environment and plate commands are 
similar to the figure environment and figure 
commands, with the word ``plate'' substituted for 
the word ``figure.''  Thus type plate captions in 
a \verb"\caption{}" command, which must appear within 
\verb"\begin{plate}" and %\linebreak 
\verb"\end{plate}" commands.  Use a \verb"\platenum{}" 
command to choose a specific plate number, and use a 
\verb"\platewidth{}" command if you need to choose a 
plate width.  Place a \verb"\setcounter{plate}{#}" 
command, where \# is the next desired equation number 
minus one, outside of a plate environment to reset 
the plate counter (this will affect plate numbers 
of all following plates).

The \verb"\plate" environment automatically generates
a second set of double-column width plate captions.

Be aware that plate width numbering commands may affect 
following figure captions and vice versa.


\subsection{Concluding the File}

The last thing in your input file is the \linebreak
\verb"\end{document}" command.  \LaTeX\ 
cannot process a file without this command.


\section{Callouts}

AGU asks authors to include callouts to indicate the first 
significant time a figure or table is mentioned (these show 
AGU where to place the graphics).  Callouts appear in the 
right margin of the original and any photocopies of your 
camera-ready manuscript.

Authors may either code callouts into the input file, 
or write them on the camera-ready pages using black ink 
(callouts will not be published).  Make sure all callouts 
appear in numerical sequence.

To create a callout, type \verb"\callout{}" around 
the text of a called-out item.  Any type within the 
curly braces will print both in the text of your 
manuscript and in a margin box to the right of your 
text.  You do not need to type called-out text twice.
See the sample.tex document for examples.


\section{Formatting Text in \LaTeX}\label{formtext}

Some text characters require special attention 
so that \LaTeX\ can properly format a file.

The following characters must be preceded by a 
backslash or \LaTeX\ will interpret them as commands:
\begin{quote}
~~~~~~~~~\$~~~\&~~~\%~~~\#~~~\_~~~\{~~~and~~~\}
\end{quote}
must be typed
\begin{center}
\begin{quote}
~~~~~~\verb"\$"~~~\verb"\&"~~~\verb"\%"~~~\verb"\#"
~~~\verb"\_"~~~\verb"\{"~~~and~~~\verb"\}".
\end{quote}
\end{center}

\LaTeX\ interprets all double quotes as closing quotes.
Therefore quotation marks must be typed as pairs of 
opening and closing single quotes, for example, 
{\tt ``quoted text.''}

Note that \LaTeX\ will not recognize greater than or
less than symbols unless they are typed within math
commands (\verb"$>$" or \verb"$<$").


\subsection{Line and Paragraph Spacing}

The ends of words and sentences are marked by white 
space.  It does not matter how many spaces are typed; 
one is as good as 100.  \LaTeX\ treats the end of a 
line in the input file as a space, not as a new line
or a new paragraph.

To create a new paragraph, leave one or more empty 
line(s) between the old paragraph and the new paragraph.


\subsubsection{Double spacing and single spacing}

Authors may wish to adjust vertical spacing within a 
preprint; for instance, double spacing text while 
single spacing tables.  To alternate between single 
and double spacing, use \verb"\singlespace" and 
\verb"\doublespace" commands.


\subsection{Line Breaks and Hyphenation}

If you need to break a line and maintain full justification,
use a \verb"\linebreak" command.  To break a line without
justification, use a double backslash command 
(\verb"\\").  Fragile arguments, such as section heads,
figure captions, or table captions, will not allow 
line breaks unless they are preceded by a {\tt \protect} 
command.  Thus, \verb"\protect\\" is appropriate for 
breaking level 1 and level 2 heads (since they are 
justified left, ragged right) and \verb"\protect\linebreak"
is appropriate for breaking level 3 and level 4 heads 
(since they have full justification).

\LaTeX\ takes care of most hyphenation automatically,
but some words are not included in the hyphenation table.
These words and other bad breaks caused by too much or
too little text on a line must be corrected by the author.
Use optional hyphens (\verb"\-") whenever possible.  When
\LaTeX\ requires a hyphen, the optional hyphen will appear
in the printout; otherwise, the word will appear unhyphenated.
(This is particularly useful when text is changed late in the
typesetting process.)

If a word requiring hyphenation occurs often in your 
manuscript, place a \hyphenation{} command in the 
preamble of your *.tex file; this will add the hyphenated 
words to {\LaTeX}'s hyphenation table.  For example, 
\verb"\hyphenation{geo-magnetism earth-quakes}".

Words that normally appear hyphenated, such as 
``author-prepared,'' may be hyphenated as usual 
(i.e., no special codes are required) but do 
not allow hyphenated words to break across lines in 
your input file, or \LaTeX\ will leave an extra space 
between the hyphen and the second half of the word.
Note that \LaTeX\ will not hyphenate a word that already 
contains a hyphen, so if ``author-prepared'' occurs 
near the end of a line you may have to provide an optional 
hyphen (\verb"author-pre\-pared") or the word ``prepared'' 
will print outside the margins.

\subsection{Bold and Italic Type}

To create boldface text, type an open curly brace,
a \verb"\bf" command, the text you want boldfaced, and
a close curly brace:  \verb"{\bf" {\bf This text is 
boldface} \verb"}".

To create italic text, type an open curly brace,
an \verb"\it" command, the text you want italicized, and
a close curly brace:  \verb"{\it" {\it This text is 
italicized} \verb"}".

\LaTeX\ automatically italicizes text within math or equation 
modes.  If you need normal (nonitalic) type within a math or
equation environment, use a \verb"\rm" command.  (For example, 
\verb"$J_{\rm Hf}t$" produces $J_{\rm Hf}(t)$).

To create boldface and italic type, use \LaTeX's 
\verb"\protect" and \verb"\boldmath" commands and place
the text inside math mode within curly braces: \linebreak 
\verb"{$\protect \boldmath" 
{$\protect \boldmath Bold\ Italic\ Text $} \verb"$}"


\section{Fonts}
AGU accepts manuscripts published in ``Computer Modern,'' 
\LaTeX's default typeface.  If authors have the ability
to select a different typeface, a ``Times Roman,'' ``Times
New Roman,'' or ``Times'' typeface is also acceptable.

Authors with Postscript printers may use AGU's \LaTeX\ 
style files to produce a Times Roman typeface.  Authors 
(or their system managers) must locate and install a dvips 
program.  Dvips is a freely redistributable PostScript 
driver for device independent files.  The program can 
be found on several different anonymous ftp sites
(including Comprehensive \TeX\ Archive Networks such as 
pip.shsu.edu or ftp.tex.ac.uk).  Once the program is 
correctly installed, authors should include the word 
``times'' in the optional argument to the document
style.
\begin{center}
\begin{quote}
\verb"\documentstyle[times,jgrga]{article}"
\end{quote}
\end{center}
Note that the {\tt times} option uses a great deal of 
memory, and that dvips is rather difficult to install.


\section{Cross-referencing}

Cross-referencing equations, tables, and figures in 
text uses ``tags,'' which are defined by the author.
Use a \verb"\label{TAG}" command to define an item 
and \verb"\ref{TAG}" command to refer to the labeled 
item.  Tags may be chosen by the author, but each 
\verb"\label" tag must match the corresponding 
\verb"\ref" tag.

Place \verb"\label" commands immediately after 
the item being labeled.  References to page numbers 
should not be made, since published page numbers 
are different from author's pages.

Table numbers are generated by the \verb"\caption" 
command, not by the \verb"\begin{}" command, so if 
you wish to \verb"\label" tables and figures, the 
\verb"\label" command should appear after the 
\verb"\caption". 

See examples of \verb"label" and \verb"ref" commands in 
the sample.tex document.

\LaTeX\ keeps track of labeling commands, cross-reference
information, and counters (such as equation, table, or 
figure counters) by maintaining an auxiliary file in the 
same working directory as the source file.  The auxiliary
file will have an extension of {\tt .aux}.  This file 
should not be deleted, since subsequent \LaTeX\ processing 
uses the auxiliary file to number items and to create 
references.  This auxiliary file mechanism makes it 
necessary to run \LaTeX\ on your input file more than 
once.  This will resolve cross-reference information 
properly.  If changes are made that affect the number 
or the placement of equations, tables, or figure captions,
\LaTeX\ will issue a warning message that advises users 
to ``rerun to get cross-references right.'' If this occurs,
you should \LaTeX\ your file again.  If the error message
appears after two successive \LaTeX\ runs, it is likely
that a reference has been made to an undefined label.

Some older versions of \LaTeX\ do not have enough
memory to run AGU's \LaTeX\ style files without pausing
at undefined references the first time a file is
{\LaTeX}ed.  Either press [Enter] each time this occurs
in the first \LaTeX\ run, or type \verb"
%%  update.tex -- August 6, 1996 version
%
%%  revisions by Amy Hendrickson, TeXnology Inc.

%   You may not need this file unless you have 
%   a very old version of LaTeX 2.09.
%
%   This file keeps old versions of LaTeX 2.09 
%   from becoming confused with \reset@font and 
%   thus crashing at every undefined label or 
%   citation.  
%
%   If you do need to include this file, 
%   type 
%%  update.tex -- August 6, 1996 version
%
%%  revisions by Amy Hendrickson, TeXnology Inc.

%   You may not need this file unless you have 
%   a very old version of LaTeX 2.09.
%
%   This file keeps old versions of LaTeX 2.09 
%   from becoming confused with \reset@font and 
%   thus crashing at every undefined label or 
%   citation.  
%
%   If you do need to include this file, 
%   type 
%%  update.tex -- August 6, 1996 version
%
%%  revisions by Amy Hendrickson, TeXnology Inc.

%   You may not need this file unless you have 
%   a very old version of LaTeX 2.09.
%
%   This file keeps old versions of LaTeX 2.09 
%   from becoming confused with \reset@font and 
%   thus crashing at every undefined label or 
%   citation.  
%
%   If you do need to include this file, 
%   type \input{update.tex} in front of your 
%   \begin{document} command in your *.tex
%   file and then continue as you would ordinarily.

\makeatletter
\let\reset@font\empty
\makeatother
 in front of your 
%   \begin{document} command in your *.tex
%   file and then continue as you would ordinarily.

\makeatletter
\let\reset@font\empty
\makeatother
 in front of your 
%   \begin{document} command in your *.tex
%   file and then continue as you would ordinarily.

\makeatletter
\let\reset@font\empty
\makeatother
"
in front of the \verb"\begin{document}" command.  The
``update.tex'' file is part of AGU's \LaTeX\ style file
package and is available on the kosmos.agu.org ftp site.


\section{Abstract Supplement (for \protect\\
\protect\boldmath $JGR$--\protect\boldmath 
$Space~Physics$ only)}

{\it JGR--Space Physics} requires authors to submit 
an abstract supplement with each camera-ready manuscript.
Abstract supplements appear in the {\it JGR--Space Physics}
subsets.  Text of abstract supplements must be 
identical to the text of the abstract published 
with your camera-ready {\it JGR--Space Physics} article. 

To generate an abstract supplement, type \linebreak
\verb"\printabstract" in front of the 
\verb"\end{document}" command.  \LaTeX\ will then 
automatically copy any text within the 
\verb"\begin{abstract}" and \verb"\end{abstract}" 
environment, format it in Abstract Supplement style, 
and print it at the end of your \LaTeX\ document.

The \verb"\printabstract" command works only with 
{\tt jgrga.sty}; it will cause all other substyles 
to crash.


\section{Additional Documentation}

The \verb"sample.tex" document is a comprehensive 
example requiring nearly all the capabilities of AGU's
\LaTeX\ style file package (in terms of commands as well
as formatting).  It also contains comments that describe
the purpose of the markup.

This user guide ({\tt\jobname.tex}) was also created 
using commands from AGU's \LaTeX\ style files, although 
it is not the best example of a scientific paper.

A number of commands described in the preceding sections
are standard \LaTeX\ commands, and the reader who is 
unfamiliar with them may refer to the \LaTeX\ manual
[{\it Lamport}, 1985] for details.  A cribsheet listing
all the \LaTeX\ commands (and some pertinent plain \TeX\
commands) with short descriptions of each is published
by the \TeX\ Users Group [{\it Botway and Biemesderfer},
1989].

Authors interested in \TeX\ (rather than \LaTeX) should 
read the {\it\TeX book} [{\it Knuth}, 1984], probably more 
than once.  There is a good deal of general information 
about typography in this source.  Also, many details of 
mathematical typography are discussed in a book by 
{\it Swanson} [1971].


\clearpage
\appendix
\section{\hspace{9pt} Appendix: Special Symbols}

AGU's \LaTeX\ style files contain an assortment
of commands for creating special symbols.  Some
are native to \LaTeX\ and others are a function
of AGU's \LaTeX\ style files.

\subsection{\LaTeX\ Symbols}
\LaTeX\ has a wide variety of special symbols for 
which markup commands have already been defined.
These range from diacritics to exotic mathematical 
operators.  Tables A1--A12 list these commands
by grouping symbols together more or less according 
to function.  Some of these symbols are primarily 
for use in text; most of them are mathematical 
symbols and can only be used in \LaTeX's 
math mode.  These tables are excerpted from the 
{\it \LaTeX\ Command Summary} [{\it Botway and 
Biemesderfer,} 1989].


\subsection{Additional Symbols}

Some of the additional symbol definitions come from the
{\it Astronomy and Astrophysics\/} package \linebreak
{[{\it Springer-Verlag}, 1989]}; some are contributions 
from individuals.  We have tried to select a tractable 
number that were useful and also somewhat difficult to 
get right:

\begin{center}
\begin{tabular}{ll@{\hspace*{3em}}ll}
\verb"\deg"    & \deg    & \verb"\sq"        & \sq \\
\verb"\sun"    & \sun    & \verb"\earth"     & \earth \\
\verb"\arcmin" & \arcmin & \verb"\arcsec"    & \arcsec \\
\verb"\fd"     & \fd     & \verb"\fh"        & \fh \\
\verb"\fm"     & \fm     & \verb"\fs"        & \fs \\
\verb"\fdg"    & \fdg    & \verb"\farcm"     & \farcm \\
\verb"\farcs"  & \farcs  & \verb"\fp"        & \fp \\
\verb"\micron" & \micron & \verb"$\lesssim$" & $\lesssim$ \\
\end{tabular}
\end{center}

Most of these commands can be used in running text as 
well as when setting mathematical expressions.  The 
\verb"\lesssim" and \verb"\gtrsim" commands can only 
be used in math mode, which is logical since they are 
relations.

When discussing atomic species, ionization levels 
can be indicated with the following command:
\begin{quote}
\verb"\ion{ELEMENT}{LEVEL}"
\end{quote}
The ionization state is specified as the second 
argument, and should be given as a numeral.
For example, \ion{Ca}{3} is specified by typing 
\verb"\ion{Ca}{3}".

\begin{references}
\reference 
   Botway, L., and C. Biemesderfer, 
   {\it \LaTeX\ Command Summary},
   \TeX\ Users Group, Providence, R. I., 1989.
 
\reference 
   Knuth, D., {\it The \TeX book}, 
   Addison-Wesley, Reading, Mass., 1984.

\reference 
   Lamport, L., 
   {\it \LaTeX: A Document Preparation System\/},
   Addison-Wesley, Reading, Mass., 1985.

\reference 
   Springer-Verlag, 
   {\it Springer-Verlag \TeX\ AA Macro Package}, 
   Springer-Verlag, New York, 1989.

\reference 
   Springer-Verlag, 
   {\it Springer-Verlag \LaTeX\ AA Macro Package}, 
   Springer-Verlag, New York, 1990.

\reference 
   Swanson, E., \, 
   {\it Mathematics \, Into \, Type}, \,
   Am. Math. Soc., Providence, R. I., 1979.
\end{references}
\end{article}

\clearpage
\setcounter{table}{0}

\begin{table}
\tablenum{A1}
\caption{Text-Mode Accents}
\tablenotetext{\null}{Note that repeated 
  column heads do not follow AGU style.} 
\begin{tabular}{cccccc}
& & & & & \\[-15pt]
\tableline
& & & & & \\[-5pt]
\multicolumn{1}{c}{Symbol} & 
\multicolumn{1}{c}{Command} & 
\multicolumn{1}{c}{Symbol} & 
\multicolumn{1}{c}{Command} & 
\multicolumn{1}{c}{Symbol} & 
\multicolumn{1}{c}{Command} \\[4pt]
\tableline
& & & & & \\[-6pt]
\`{o}         & \verb"\`{o}"        & \={o}         & 
\verb"\={o}"  & \t{oo}              & \verb"\t{oo}" \\
\'{o}         & \verb"\'{o}"        & \.{o}         & 
\verb"\.{o}"  & \c{o}               & \verb"\c{o}"  \\
\^{o}         & \verb"\^{o}"        & \u{o}         &
\verb"\u{o}"  & \d{o}               & \verb"\d{o}"  \\
\"{o}         & \verb"\""\verb"{o}" & \v{o}         & 
\verb"\v{o}"  & \b{o}               & \verb"\b{o}"  \\
\~{o}         & \verb"\~{o}"        & \H{o}         &
\verb"\H{o}"  &                     &               \\[4pt]
\tableline
& & & \\[-3pt]
\end{tabular}
\end{table}

\vspace{3pc}

\begin{table}
\caption{National Symbols}
\begin{tabular}{ccclcl}
& & & & & \\[-15pt]
\tableline
& & & & & \\[-5pt]
\multicolumn{1}{c}{Symbol}  & 
\multicolumn{1}{c}{Command} & 
\multicolumn{1}{c}{Symbol}  & 
\multicolumn{1}{l}{Command} & 
\multicolumn{1}{c}{Symbol}  & 
\multicolumn{1}{l}{Command} \\[4pt]
\tableline
& & & & & \\[-6pt]
\oe            & \verb"\oe" & \aa           & 
\verb"   \aa"  & \l         & \verb"   \l"  \\
\OE            & \verb"\OE" & \AA           & 
\verb"   \AA"  & \L         & \verb"   \L"  \\
\ae            & \verb"\ae" & \o            & 
\verb"   \o"   & \ss        & \verb"   \ss" \\
\AE            & \verb"\AE" & \O            & 
\verb"   \O"   &            &               \\[4pt]
\tableline
& & & \\[-6pt]
\end{tabular}
\end{table}

\vspace{3pc}

\begin{table}
\caption{Miscellaneous Symbols}
\begin{tabular}{clcccl}
& & & & & \\[-15pt]
\tableline
& & & & & \\[-5pt]
\multicolumn{1}{c}{Symbol} & 
\multicolumn{1}{l}{Command} & 
\multicolumn{1}{c}{Symbol} & 
\multicolumn{1}{c}{Command} & 
\multicolumn{1}{c}{Symbol} & 
\multicolumn{1}{l}{Command} \\[4pt]
\tableline
& & & & & \\[-6pt]
\dag        & \verb"\dag"  & \S                & 
\verb"\S"   & \copyright   & \verb"\copyright" \\
\ddag       & \verb"\ddag" & \P                & 
\verb"\P"   & \pounds      & \verb"\pounds"    \\
\#          & \verb"\#"    & \$                & 
\verb"\$"   & \%           & \verb"\%"         \\
\&          & \verb"\&"    & \_                & 
\verb"\_"   &              &                   \\
\{          & \verb"\{"    & \}                & 
\verb"\}"   &              &                   \\[4pt]
\tableline
& & & & & \\[-6pt]
\end{tabular}
\end{table}

\vspace{3pc}

\begin{table}
\caption{Math-Mode Accents}
\begin{tabular}{cl@{\hspace{4em}}cl}
& & & \\[-15pt]
\tableline
& & & \\[-5pt]
\multicolumn{1}{c}{Symbol} & 
\multicolumn{1}{l}{Command} & 
\multicolumn{1}{c}{Symbol} & 
\multicolumn{1}{l}{Command} \\[4pt]
\tableline
& & & \\[-6pt]
$\hat{a}$   & \verb"\hat{a}"   & 
$\dot{a}$   & \verb"\dot{a}"   \\
$\check{a}$ & \verb"\check{a}" & 
$\ddot{a}$  & \verb"\ddot{a}"  \\
$\tilde{a}$ & \verb"\tilde{a}" & 
$\breve{a}$ & \verb"\breve{a}" \\
$\acute{a}$ & \verb"\acute{a}" & 
$\bar{a}$   & \verb"\bar{a}"   \\
$\grave{a}$ & \verb"\grave{a}" & 
$\vec{a}$   & \verb"\vec{a}"   \\[4pt]
\tableline
& & & \\[-6pt]
\end{tabular}
\end{table}

\clearpage

\begin{table}
\caption{Greek Letters (Math Mode)}
\begin{tabular}{cl@{\hspace{3em}}cl}
& & & \\[-15pt]
\tableline
& & & \\[-5pt]
\multicolumn{1}{c}{Symbol} & 
\multicolumn{1}{l}{Command} & 
\multicolumn{1}{c}{Symbol} & 
\multicolumn{1}{l}{Command} \\[4pt]
\tableline
& & & \\[-6pt]
$\alpha$      & \verb"\alpha"      & 
$\nu$         & \verb"\nu"         \\
$\beta$       & \verb"\beta"       & 
$\xi$         & \verb"\xi"         \\
$\gamma$      & \verb"\gamma"      & 
$o$           & \verb"o"           \\
$\delta$      & \verb"\delta"      & 
$\pi$         & \verb"\pi"         \\
$\epsilon$    & \verb"\epsilon"    & 
$\rho$        & \verb"\rho"        \\
$\zeta$       & \verb"\zeta"       & 
$\sigma$      & \verb"\sigma"      \\
$\eta$        & \verb"\eta"        & 
$\tau$        & \verb"\tau"        \\
$\theta$      & \verb"\theta"      & 
$\upsilon$    & \verb"\upsilon"    \\
$\iota$       & \verb"\iota"       & 
$\phi$        & \verb"\phi"        \\
$\kappa$      & \verb"\kappa"      & 
$\chi$        & \verb"\chi"        \\
$\lambda$     & \verb"\lambda"     & 
$\psi$        & \verb"\psi"        \\
$\mu$         & \verb"\mu"         & 
$\omega$      & \verb"\omega"      \vspace{1em}\\
$\varepsilon$ & \verb"\varepsilon" & 
$\varsigma$   & \verb"\varsigma"   \\
$\vartheta$   & \verb"\vartheta"   & 
$\varphi$     & \verb"\varphi"     \\
$\varrho$     & \verb"\varrho"     & 
              & \vspace{1em}       \\
$\Gamma$      & \verb"\Gamma"      & 
$\Sigma$      & \verb"\Sigma"      \\
$\Delta$      & \verb"\Delta"      & 
$\Upsilon$    & \verb"\Upsilon"    \\
$\Theta$      & \verb"\Theta"      & 
$\Phi$        & \verb"\Phi"        \\
$\Lambda$     & \verb"\Lambda"     &
$\Psi$        & \verb"\Psi"        \\
$\Xi$         & \verb"\Xi"         & 
$\Omega$      & \verb"\Omega"      \\
$\Pi$         & \verb"\Pi"         & 
              &                    \\[4pt]
\tableline
& & & \\[-6pt]
\end{tabular}
\end{table}

\vspace{3pc}

\begin{table}
\caption{Binary Operations (Math Mode)}
\begin{tabular}{cl@{\hspace{3em}}cl}
& & & \\[-15pt]
\tableline
& & & \\[-5pt]
\multicolumn{1}{c}{Symbol} & 
\multicolumn{1}{l}{Command} & 
\multicolumn{1}{c}{Symbol} & 
\multicolumn{1}{l}{Command} \\[4pt]
\tableline
& & & \\[-6pt]
$\pm$              & \verb"\pm"              & 
$\cap$             & \verb"\cap"             \\
$\mp$              & \verb"\mp"              & 
$\cup$             & \verb"\cup"             \\
$\setminus$        & \verb"\setminus"        & 
$\uplus$           & \verb"\uplus"           \\
$\cdot$            & \verb"\cdot"            & 
$\sqcap$           & \verb"\sqcap"           \\
$\times$           & \verb"\times"           & 
$\sqcup$           & \verb"\sqcup"           \\
$\ast$             & \verb"\ast"             & 
$\triangleleft$    & \verb"\triangleleft"    \\
$\star$            & \verb"\star"            & 
$\triangleright$   & \verb"\triangleright"   \\
$\diamond$         & \verb"\diamond"         & 
$\wr$              & \verb"\wr"              \\
$\circ$            & \verb"\circ"            & 
$\bigcirc$         & \verb"\bigcirc"         \\
$\bullet$          & \verb"\bullet"          & 
$\bigtriangleup$   & \verb"\bigtriangleup"   \\
$\div$             & \verb"\div"             & 
$\bigtriangledown$ & \verb"\bigtriangledown" \\
$\lhd$             & \verb"\lhd"             & 
$\rhd$             & \verb"\rhd"             \\
$\vee$             & \verb"\vee"             & 
$\odot$            & \verb"\odot"            \\
$\wedge$           & \verb"\wedge"           & 
$\dagger$          & \verb"\dagger"          \\
$\oplus$           & \verb"\oplus"           & 
$\ddagger$         & \verb"\ddagger"         \\
$\ominus$          & \verb"\ominus"          & 
$\amalg$           & \verb"\amalg"           \\
$\otimes$          & \verb"\otimes"          & 
$\unlhd$           & \verb"\unlhd"           \\
$\oslash$          & \verb"\oslash"          & 
$\unrhd$           & \verb"\unrhd"           \\[4pt]
\tableline
& & & \\[-6pt]
\end{tabular}
\end{table}

\vspace{3pc}

\begin{table}
\caption{Relations (Math Mode)}
\begin{tabular}{cl@{\hspace{4em}}cl}
& & & \\[-15pt]
\tableline
& & & \\[-5pt]
\multicolumn{1}{c}{Symbol} & 
\multicolumn{1}{l}{Command} & 
\multicolumn{1}{c}{Symbol} & 
\multicolumn{1}{l}{Command} \\[4pt]
\tableline
& & & \\[-6pt]
$\leq$        & \verb"\leq"        & 
$\geq$        & \verb"\geq"        \\
$\prec$       & \verb"\prec"       & 
$\succ$       & \verb"\succ"       \\
$\preceq$     & \verb"\preceq"     & 
$\succeq$     & \verb"\succeq"     \\
$\ll$         & \verb"\ll"         & 
$\gg$         & \verb"\gg"         \\
$\subset$     & \verb"\subset"     & 
$\supset$     & \verb"\supset"     \\
$\subseteq$   & \verb"\subseteq"   & 
$\supseteq$   & \verb"\supseteq"   \\
$\sqsubset$   & \verb"\sqsubset"   & 
$\sqsupset$   & \verb"\sqsupset"   \\
$\sqsubseteq$ & \verb"\sqsubseteq" & 
$\sqsupseteq$ & \verb"\sqsupseteq" \\
$\in$         & \verb"\in"         & 
$\ni$         & \verb"\ni"         \\
$\vdash$      & \verb"\vdash"      & 
$\dashv$      & \verb"\dashv"      \\
$\smile$      & \verb"\smile"      & 
$\mid$        & \verb"\mid"        \\
$\frown$      & \verb"\frown"      & 
$\parallel$   & \verb"\parallel"   \\
$\neq$        & \verb"\neq"        & 
$\perp$       & \verb"\perp"       \\
$\equiv$      & \verb"\equiv"      & 
$\cong$       & \verb"\cong"       \\
$\sim$        & \verb"\sim"        & 
$\bowtie$     & \verb"\bowtie"     \\
$\simeq$      & \verb"\simeq"      & 
$\propto$     & \verb"\propto"     \\
$\asymp$      & \verb"\asymp"      & 
$\models$     & \verb"\models"     \\
$\approx$     & \verb"\approx"     & 
$\doteq$      & \verb"\doteq"      \\
              &                    & 
$\Join$       & \verb"\Join"       \\[4pt]
\tableline
& & & \\[-6pt]
\end{tabular}
\end{table}

\vspace{3pc}

\begin{table}
\caption{Variable-Sized Symbols (Math Mode)}
\begin{tabular}{ccl@{\hspace{2em}}ccl}
& & & \\[-15pt]
\tableline
& & & \\[-5pt]
\multicolumn{1}{c}{Symbol} & 
\multicolumn{1}{c}{Display Style} & 
\multicolumn{1}{l}{Command} & 
\multicolumn{1}{c}{Symbol} & 
\multicolumn{1}{c}{Display Style} &
\multicolumn{1}{l}{Command}  \\[4pt]
\tableline
& & & \\[-6pt]
$\sum$                     & $\displaystyle \sum$       &
$\hbox{\verb"\sum"}$       & $\bigcap$                  &
$\displaystyle \bigcap$    & $\hbox{\verb"\bigcap"}$
\vspace{4pt}\\ 
$\prod$                    & $\displaystyle \prod$      &
$\hbox{\verb"\prod"}$      & $\bigcup$                  &
$\displaystyle \bigcup$    & $\hbox{\verb"\bigcup"}$
\vspace{4pt}\\ 
$\coprod$                  & $\displaystyle \coprod$    &
$\hbox{\verb"\coprod"}$    & $\bigsqcup$                &
$\displaystyle \bigsqcup$  & $\hbox{\verb"\bigsqcup"}$
\vspace{4pt}\\ 
$\int$                     & $\displaystyle \int$       &
$\hbox{\verb"\int"}$       & $\bigvee$                  &
$\displaystyle \bigvee$    & $\hbox{\verb"\bigvee"}$
\vspace{4pt}\\ 
$\oint$                    & $\displaystyle \oint$      &
$\hbox{\verb"\oint"}$      & $\bigwedge$                &
$\displaystyle \bigwedge$  & $\hbox{\verb"\bigwedge"}$
\vspace{4pt}\\ 
$\bigodot$                 & $\displaystyle \bigodot$   &
$\hbox{\verb"\bigodot"}$   & $\bigotimes$               &
$\displaystyle \bigotimes$ & $\hbox{\verb"\bigotimes"}$ 
\vspace{4pt}\\ 
$\bigoplus$                & $\displaystyle \bigoplus$  &
$\hbox{\verb"\bigoplus"}$  & $\biguplus$                &
$\displaystyle \biguplus$  & $\hbox{\verb"\biguplus"}$  \\[4pt]
\tableline
& & & \\[-6pt]
\end{tabular}
\end{table}

\vspace{3pc}

\begin{table}
\caption{Delimiters (Math Mode)}
\begin{tabular}{cl@{\hspace{2em}}cl}
& & & \\[-15pt]
\tableline
& & & \\[-5pt]
\multicolumn{1}{l}{Symbol} & 
\multicolumn{1}{l}{Command} & 
\multicolumn{1}{l}{Symbol} & 
\multicolumn{1}{l}{Command} \\[4pt]
\tableline
& & & \\[-6pt]
$($            & \verb"("            & 
$)$            & \verb")"            \\
$[$            & \verb"["            & 
$]$            & \verb"]"            \\
$\{$           & \verb"\{"           & 
$\}$           & \verb"\}"           \\
$\lfloor$      & \verb"\lfloor"      & 
$\rfloor$      & \verb"\rfloor"      \\
$\lceil$       & \verb"\lceil"       &
$\rceil$       & \verb"\rceil"       \\
$\langle$      & \verb"\langle"      & 
$\rangle$      & \verb"\rangle"      \\
$/$            & \verb"/"            & 
$\backslash$   & \verb"\backslash"   \\
$\vert$        & \verb"\vert"        & 
$\Vert$        & \verb"\Vert"        \\
$\uparrow$     & \verb"\uparrow"     & 
$\Uparrow$     & \verb"\Uparrow"     \\
$\downarrow$   & \verb"\downarrow"   & 
$\Downarrow$   & \verb"\Downarrow"   \\
$\updownarrow$ & \verb"\updownarrow" & 
$\Updownarrow$ & \verb"\Updownarrow" \\[4pt]
\tableline
& & & \\[-6pt]
\end{tabular}
\end{table}

\vspace{3pc}

\begin{table}
\caption{Function Names (Math Mode)}
\begin{tabular}{l@{\hspace{2pc}}l@{\hspace{2pc}}
                l@{\hspace{2pc}}l@{\hspace{2pc}}}
& & & \\[-15pt]
\tableline
& & & \\[-6pt]
\verb"\arccos" &   \verb"\csc" &
\verb"\ker"    &  \verb"\min"  \\
\verb"\arcsin" &   \verb"\deg" &
\verb"\lg"     &  \verb"\Pr"   \\
\verb"\arctan" &   \verb"\det" &
\verb"\lim"    &  \verb"\sec"  \\
\verb"\arg"    &   \verb"\dim" &
\verb"\liminf" &  \verb"\sin"  \\
\verb"\cos"    &   \verb"\exp" &
\verb"\limsup" &  \verb"\sinh" \\
\verb"\cosh"   &   \verb"\gcd" &
\verb"\ln"     &  \verb"\sup"  \\
\verb"\cot"    &   \verb"\hom" &
\verb"\log"    &  \verb"\tan"  \\
\verb"\coth"   &   \verb"\inf" &
\verb"\max"    &  \verb"\tanh" \\[4pt]
\tableline
& & & \\[-6pt]
\end{tabular}
\end{table}

\vspace{6pc}

\begin{table}
\caption{Arrows (Math Mode)}
\begin{tabular}{clcl}
& & & \\[-15pt]
\tableline
& & & \\[-5pt]
\multicolumn{1}{c}{Symbol} &
\multicolumn{1}{l}{Command} & 
\multicolumn{1}{c}{Symbol} & 
\multicolumn{1}{l}{Command} \\[4pt]
\tableline
& & & \\[-6pt]
$\leftarrow$          & \verb"\leftarrow"          & 
$\longleftarrow$      & \verb"\longleftarrow"      \\
$\Leftarrow$          & \verb"\Leftarrow"          & 
$\Longleftarrow$      & \verb"\Longleftarrow"      \\
$\rightarrow$         & \verb"\rightarrow"         & 
$\longrightarrow$     & \verb"\longrightarrow"     \\
$\Rightarrow$         & \verb"\Rightarrow"         & 
$\Longrightarrow$     & \verb"\Longrightarrow"     \\
$\leftrightarrow$     & \verb"\leftrightarrow"     & 
$\longleftrightarrow$ & \verb"\longleftrightarrow" \\
$\Leftrightarrow$     & \verb"\Leftrightarrow"     & 
$\Longleftrightarrow$ & \verb"\Longleftrightarrow" \\
$\mapsto$             & \verb"\mapsto"             & 
$\longmapsto$         & \verb"\longmapsto"         \\ 
$\hookleftarrow$      & \verb"\hookleftarrow"      & 
$\hookrightarrow$     & \verb"\hookrightarrow"     \\
$\leftharpoonup$      & \verb"\leftharpoonup"      & 
$\rightharpoonup$     & \verb"\rightharpoonup"     \\
$\leftharpoondown$    & \verb"\leftharpoondown"    & 
$\rightharpoondown$   & \verb"\rightharpoondown"   \\
$\rightleftharpoons$  & \verb"\rightleftharpoons"  & 
$\leadsto$            & \verb"\leadsto"            \\
$\uparrow$            & \verb"\uparrow"            & 
$\Updownarrow$        & \verb"\Updownarrow"        \\
$\Uparrow$            & \verb"\Uparrow"            & 
$\nearrow$            & \verb"\nearrow"            \\
$\downarrow$          & \verb"\downarrow"          & 
$\searrow$            & \verb"\searrow"            \\
$\Downarrow$          & \verb"\Downarrow"          &
$\swarrow$            & \verb"\swarrow"            \\
$\updownarrow$        & \verb"\updownarrow"        & 
$\nwarrow$            & \verb"\nwarrow"            \\[4pt]
\tableline
& & & \\[-6pt]
\end{tabular}
\end{table}

\vspace{3pc}

\begin{table}
\caption{Miscellaneous Symbols (Math Mode)}
\begin{tabular}{cl@{\hspace{3em}}cl}
& & & \\[-15pt]
\tableline
& & & \\[-5pt]
\multicolumn{1}{c}{Symbol} & 
\multicolumn{1}{l}{Command} & 
\multicolumn{1}{c}{Symbol} & 
\multicolumn{1}{l}{Command} \\[4pt]
\tableline
& & & \\[-6pt]
$\aleph$       & \verb"\aleph"       & 
$\prime$       & \verb"\prime"       \\
$\hbar$        & \verb"\hbar"        & 
$\emptyset$    & \verb"\emptyset"    \\
$\imath$       & \verb"\imath"       & 
$\nabla$       & \verb"\nabla"       \\
$\jmath$       & \verb"\jmath"       & 
$\surd$        & \verb"\surd"        \\
$\ell$         & \verb"\ell"         & 
$\top$         & \verb"\top"         \\
$\wp$          & \verb"\wp"          & 
$\bot$         & \verb"\bot"         \\
$\Re$          & \verb"\Re"          & 
$\|$           & \verb"\|"           \\
$\Im$          & \verb"\Im"          & 
$\angle$       & \verb"\angle"       \\
$\partial$     & \verb"\partial"     & 
$\triangle$    & \verb"\triangle"    \\
$\infty$       & \verb"\infty"       & 
$\backslash$   & \verb"\backslash"   \\
$\Box$         & \verb"\Box"         & 
$\Diamond$     & \verb"\Diamond"    \\
$\forall$      & \verb"\forall"      & 
$\sharp$       & \verb"\sharp"       \\
$\exists$      & \verb"\exists"      & 
$\clubsuit$    & \verb"\clubsuit"    \\
$\neg$         & \verb"\neg"         & 
$\diamondsuit$ & \verb"\diamondsuit" \\
$\flat$        & \verb"\flat"        & 
$\heartsuit$   & \verb"\heartsuit"   \\
$\natural$     & \verb"\natural"     & 
$\spadesuit$   & \verb"\spadesuit"   \\
$\mho$         & \verb"\mho"         & 
               &                     \\[4pt]
\tableline
& & & \\[-6pt]
\end{tabular}
\end{table}

\end{document}
